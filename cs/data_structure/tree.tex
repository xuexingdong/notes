\documentclass{article}

\usepackage{ctex}

\title{二叉树}
\author{Xingdong Xue}
\date{Mar. 2020}

\begin{document}
\maketitle

\section{树}
\subsection{树的形成}
树结合了向量和列表,在\textbf{插入}和\textbf{查找}上都有不错的效率,是一种\textbf{半线性}的数据结构。

\subsection{树的应用}
\begin{itemize}
  \item 层次关系的数据表示
  \item 表达式
  \item 文件系统结构
\end{itemize}

\subsection{有根树}

树是一种特殊的图,把图中一顶点作为根(root)后,则图可以称作有根树(rooted tree)。

对任意几个有根树进行合并的结果一样是有根树。

相对于树T,它的子树记做$T_i = subtree(r_i)$。

\subsection{有序树}
了解以下概念:

\begin{itemize}
  \item 孩子(child)
  \item 兄弟(sibling)
  \item 父亲(parent)
  \item 祖先(ancestor)
  \item 度(degree):一个顶点孩子的数目。
\end{itemize}

可以用顶点数n作为复杂度的参照,因为有以下公式:
$$edge = n-1 = \sum_{r \in V}{degree(r)} $$

\subsection{路径和环路}

如果k+1个点通过k条边相邻,则构成一条路径(path),也称通路,路径长度 = k。

如果一个路径的首尾顶点相同,则成为环路(cycle/loop)。

\subsection{连通和无环}

如果任意两顶点之间均有路径,称作连通图(connected);不含环路,称为无环图(acyclic)。

树是一种\textbf{无环连通图},也是原图的\textbf{极小连通图}和\textbf{极大无环图}。

\subsection{深度和层次}

不引起歧义的情况下,路径,顶点,子树可以相互指代,即:
$$path(v) \sim v \sim subtree(v)$$

有定义v的深度:
$$depth(v) = \left|x\right|$$

在$path(v)$上的顶点,均称作v的祖先(ancestor),v是它们的后代(descendent),同时除v自身外,是真(proper)祖先/后代。

半线性:在任一深度,祖先(前驱)唯一,后代(后继)不唯一。
根顶点是所有顶点的公共祖先,深度为0。

没有后代的顶点被称为叶子(leaf)。

叶子深度中的最大者,称作树的高度(height),且有
$$height(v) = height(subtree(v))$$

特别地,定义空树的高度是-1。

$$depth(v) + height(v) \leq height(T)$$

\section{树的表示}
\subsection{树的接口}
\begin{enumerate}
  \item root():获取根节点
  \item parent():获取父节点
  \item firstChild():获取第一个子节点
  \item nextSibling():兄弟
  \item insert(i, e):将e作为第i个孩子插入
  \item remove(i):删除第i个节点
  \item traverse():遍历
\end{enumerate}

\subsection{父亲}
\begin{itemize}
  \item rank:序号
  \item data:数据
  \item parent:父节点序号,根节点为-1。
\end{itemize}

时间复杂度在寻找父亲与根节点时是$O(1)$,查找孩子兄弟是$O(n)$。

\subsection{孩子}
\begin{itemize}
  \item rank:序号
  \item data:数据
  \item children:孩子列表
\end{itemize}

时间复杂度在寻找孩子节点时是$O(1)$,查找父亲是$O(n)$。

\subsection{父亲+孩子}
\begin{itemize}
  \item rank:序号
  \item data:数据
  \item parent:父亲
  \item children:孩子列表
\end{itemize}

时间复杂度在寻找孩子和父亲节点时都是$O(1)$,但各个节点的children可能要保留n个引用。

\subsection{长子+兄弟}
\begin{itemize}
  \item rank:序号
  \item data:数据
  \item parent:父亲
  \item firstChild:长子
  \item nextSibling:下一个兄弟
\end{itemize}

时间复杂度在寻找孩子和父亲节点时都是$O(1)$,且各个节点的children最多2个引用。

\section{二叉树}
\subsection{概述}

每个节点度数不超过2的树,成为二叉树(binary tree)。

可以划分为左子树和右子树,默认左子树在前。

一些规律:
\begin{enumerate}
  \item 深度为k的节点,至多$2^k$个。
  \item 含n各节点,高度为h的二叉树中,$h<n<2^{h+1} \ sim 2 \times 2^h$。
  \item $n=h+1$时,为单链;$n=2^{h+1}$时,为满二叉树(full binary tree)。
  \item 高度增长的缓慢,宽度增长的迅速。
\end{enumerate}

\subsection{真二叉树}

每个节点的度数均为偶数的二叉树为真二叉树。
可以通过为单孩子的节点添加虚拟孩子来实现真二叉树。

\subsection{描述多叉树}

多叉树可以通过长子-兄弟表示法并进行旋转后,得到一颗二叉树。

\section{二叉树实现}
\subsection{BinNode模板类}
\begin{itemize}
  \item parent
  \item lChild
  \item rChile
  \item height
  \item npl:左式堆
  \item color:红黑树
\end{itemize}

\subsection{高度更新}
高度发生变化是因为左子树或右子树高度发生变化,树的高度是左子树或右子树高度中的最大者+1。

更新高度时,循环更新x的父级节点的高度直到根节点。

复杂度为节点的深度。

\subsection{节点插入}

\begin{enumerate}
  \item 增加树的长度
  \item 挂上节点
  \item 更新全树的高度
\end{enumerate}

\subsection{遍历}

\begin{itemize}
  \item 先序(preorder)
  \item 中序(inorder)
  \item 后序(postorder)
  \item 层次(广度):自上而下,自作而右
\end{itemize}

\subsection{递归遍历}
\begin{enumerate}
  \item 如果节点为空,则返回
  \item 访问当前节点
  \item 遍历节点的左孩子节点
  \item 遍历节点的右孩子节点
\end{enumerate}

\subsection{迭代先序遍历1}

如果递归形式的调用都出现在函数的末尾,则成为尾递归,尾递归通常可以转换为迭代的形式进行优化。
\begin{enumerate}
  \item 声明一个空栈并让根节点入栈
  \item 弹出栈元素并访问
  \item 将右孩子入栈
  \item 将左孩子入栈
  \item 循环步骤234直到栈为空
\end{enumerate}

\subsection{迭代先序遍历2}

先序遍历的逻辑可以简化为先无限从左子树往下进行访问,然后再从右子树进行倒序回溯。
假定某棵子树的所有左子节点为$L_1 \cdots L_n$,与节点平级的右子树为$T_1 \cdots T_n$,则二叉树的遍历顺序可以简单表示为
$$L_1 \rightarrow \cdots L_n \rightarrow T_n \rightarrow \cdots T_1$$

\begin{enumerate}
  \item 声明一个空栈
  \item \begin{enumerate}
          \item 访问当前节点
          \item 将该节点右孩子(右子树)入栈
          \item 迭代到下一个左子树
          \item 循环直到左侧链被遍历完毕
        \end{enumerate}
  \item 如果栈为空,则退出循环
  \item 不为空则弹出下一棵右子树的根
  \item 循环2至4步
\end{enumerate}

\subsection{中序遍历}

中序遍历的逻辑可以简化为先访问最深的左子树,然后再访问对应的右子树,然后将这科子树剔除,递归的。
假定某棵子树的所有左子节点为$L_1 \cdots L_n$,与节点平级的右子树为$T_1 \cdots T_n$,则二叉树的遍历顺序可以简单表示为
$$L_n \rightarrow T_n \rightarrow L_{n-1} \rightarrow T_{n-1} \rightarrow \cdots L_1 \rightarrow T_1$$

\begin{enumerate}
  \item 声明一个空栈
  \item \begin{enumerate}
          \item 将该节点左孩子(左子树)入栈
          \item 迭代到下一个左子树
          \item 循环直到左侧链被遍历完毕
        \end{enumerate}
  \item 如果栈为空,则退出循环
  \item 不为空则弹出栈内节点
  \item 访问新节点
  \item 转向右子树(注意为空)
  \item 循环2至6步
\end{enumerate}

复杂度不为$O(n^2)$,因为循环中对n个节点进行了分摊。

\subsection{层次遍历}
因为先序,中序,后序遍历都有下一层节点先于上一层节点被访问的情况,所以无法实现需求。
层次遍历是严格意义上的先进先出,自热而然想到队列的数据结构。

\begin{enumerate}
  \item 声明一个空队列
  \item 根节点入队
  \item \begin{enumerate}
          \item 从队列里取一个节点
          \item 访问该节点
          \item 左孩子入队
          \item 右孩子入队
        \end{enumerate}
  \item 循环执行3直至队列为空
\end{enumerate}

\subsection{重构}
根据树的序列还原树的结构

先序或后序加上中序可以确定一棵二叉树。

[先序+后序]*真可以确定一棵二叉树。

\end{document}