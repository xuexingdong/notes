\documentclass{article}

\usepackage{ctex}
\usepackage{amsmath}
\usepackage{url}
\newtheorem{exmp}{Example}

\title{MIT18.01 Single Variable Calculus}
\author{Xingdong Xue}
\date{Nov. 2019}

\newcommand\limitx[1]{\lim_{x \to #1}}
\newcommand\limitdeltaxzero{\lim_{\Delta x \to 0}}

\newcommand\defint[3]{\int_{#1}^{#2}#3dx}
\newcommand\upperxdefint[2]{\int_{#1}^x{#2}dt}
\newcommand\improperint[1]{\int_{#1}^\infty}

\begin{document}
\maketitle

\section{Derivatives}
\subsection{What is Derivative}

\textbf{Geometric Interpretation}

Derivative is the slope of the line tangent to the graph of $f(x)$.

Tangent line: The limit of the secant line (a line drawn between two points on the graph) as the distance between the two points goes to zero.
$$\limitdeltaxzero \frac{\Delta f}{\Delta x} = \limitdeltaxzero \underbrace{\frac{f(x_0+\Delta x)-f(x_0)}{\Delta x}}_{difference\ quotient} = \underbrace{f'(x_0)}_{derivative\ of\ f\ at\ x_0}$$

\textbf{Physical Interpretation}

It's a rate of change, $\frac{\Delta y}{\Delta x}$ is average change, while $x \rightarrow 0$, it becomes instantaneous rate $\frac{dy}{dx}$.

\begin{exmp}
$f(x) = \frac{1}{x}$, find $f'(x)$.
  \begin{align*}
    f'(x) &= \limitdeltaxzero\frac{\frac{1}{x+\Delta x} - \frac{1}{x}}{\Delta x} \\
          &= \limitdeltaxzero\frac{1}{\Delta x} \times \frac{x - (x+\Delta x)}{(x+\Delta x)x} \\
          &= \limitdeltaxzero\frac{1}{\Delta x} \times \frac{-\Delta x}{(x+\Delta x)x} \\
          &= \limitdeltaxzero-\frac{1}{x^2+x\Delta x} \\
          &= \limitdeltaxzero \\
          &= -\frac{1}{x^2}
  \end{align*}
\end{exmp}

\begin{exmp}
  $f(x) = \sin x$, find $f'(x)$.
  \begin{align*}
    f'(x) &= \limitdeltaxzero\frac{\sin (x+\Delta x) - \sin x}{\Delta x} \\
          &= \limitdeltaxzero\frac{1}{\Delta x}(\sin x \cos \Delta x + \sin \Delta x\cos x - \sin x) \\
          &= \limitdeltaxzero\frac{\sin x(\cos \Delta x -1)}{\Delta x} + \frac{\sin \Delta x \cos x}{\Delta x} \\
          &= 0 + \cos x \\
          &= \cos x
  \end{align*}
\end{exmp}

\textbf{Notation}
$$\frac{\Delta f}{\Delta x} \rightarrow f'(x) \quad (Newton's \  Notation)$$
$$\frac{\Delta d}{\Delta x} \rightarrow \frac{dy}{dx} \quad (Leibniz's \  Notation)$$
$$\frac{df}{dx}, f', D\ f$$

\subsection{Limits and Continuity}

\textbf{Easy Limits}

Just plug in the limit to evaluate.
$$\limitdeltaxzero x + 1 = 1$$

\textbf{Continuity}

Left-hand Limit: $\limitx{x_0^-}$

Right-hand Limit: $\limitx{x_0^+}$

$f(x)$ is continuous at $x_0$ when
$$\limitx{x_0} f(x) = f(x_0)$$

It equals to
\begin{enumerate}
\item $\limitx{x_0}f(x_0)$ exists
\item $f(x_0)$ is defined
\item $\limitx{x_0^-} = \limitx{x_0^+}$
\end{enumerate}

\textbf{Discontinuity}

\begin{enumerate}
\item Removable Discontinuity: $\limitx{x_0^-} = \limitx{x_0^+}$.
\item Jump Discontinuity: $\limitx{x_0^-} \not= \limitx{x_0^+}$.
\item Infinite Discontinuity: $\limitx{x_0^-} = \pm\infty$, $\limitx{x_0^+} = \pm\infty$.
\item Other Discontinuity: $\sin{\frac{1}{x}}$, no left or right limit.
\end{enumerate}

\textbf{Differentiable}

Left differential and right differential exist and equal.

\textbf{Differentiable Implies Continuous}

$$\limitx{x_0}{(f(x)-f(x_0))} = \limitx{x_0}\left[{\frac{f(x)-f(x_0)}{x-x_0}}\right](x-x_0) = f'(x_0) \cdot x_0 = 0$$

But Not vice versa.
such as $y = x^{\frac{1}{3}}$ and $y = \left|x\right|$.

\subsection{Differentiate Fomulas}
\textbf{Specific}

$$(\sin x)' = \cos x$$
$$(\cos x)' = -\sin x$$

\textbf{General}

\begin{enumerate}
\item Product Rule: $(uv)' = u'v + uv'$.
\item Quotient Rule: $(\frac{u}{v})' = \frac{u'v-uv'}{v^2}(v \not= 0)$.
\item Chain Rule: use new variable names
\end{enumerate}

\subsection{Higher Derivatives}
$$f''(x) = D^2f = \frac{d^2f}{dx^2}$$

\subsection{Implicit Differentiation and Inverses}

\textbf{Implicit Differentiation}
\begin{gather*}
x^2 + y^2 = 1 \\
2x + 2y\frac{dy}{dx} = 0
\end{gather*}

\textbf{Inverse Functions}

if $f(x) = y$ and $g(y) = x$, then we call $g$ the inverse function of $f$, also $f^{-1}$.

$f^{-1}$ as the graph of $f$ reflected about the line $y = x$.

\subsection{Exponential and Log, Logarithmic Differentiation, Hyperbolic Functions}
\textbf{How to Find e?}
\begin{enumerate}
  \item let $M(a) = \limitx{0}\frac{a^{\Delta x} - 1}{\Delta x}$
  \item so $\frac{d}{dx}a^x = M(a)a^x$
  \item let $M(e) = 1$
  \item $\frac{d}{dx}e^x = e^x$
  \item $\limitx{0}\frac{e^x - 1}{x} = 1$
\end{enumerate}

\textbf{An Important Limit About e}
$$\limitx{\infty}(1 + \frac{1}{x})^x = e$$

\textbf{How to Differentiate $a^x$?}

\begin{itemize}
  \item $a^x = e^{x\ln a}$
  \item use logarithmic differentiation
\end{itemize}

\textbf{Hyperbolic Sine and Cosine}

$$\sinh x = \frac{e^x - e^{-x}}{2}$$
$$\cosh x = \frac{e^x + e^{-x}}{2}$$
$$\cosh ^2(x) - \sinh ^2(x) = 1$$

\section{Applications of Differentiation}
\subsection{Linear Approximation}
$$f(x) \approx f(x_0) + f'(x_0)(x-x_0)$$

geometric significance: the best fit straight line of a function

\subsection{Quadratic Approximation}
$$f(x) \approx f'(x_0)(x-x_0) + \frac{f''(x_0)}{2}(x-x_0)^2$$
more elaborate than linear approximation \\
geometric significance: the best fit parabola of a function

\subsection{Curve Sketching}
\begin{itemize}
  \item $f' > 0$, function is increasing
  \item $f' < 0$, function is decreasing
  \item $f' = 0$, $x_0$ is critical point, $y$ is the critical value.
  \item $f'' > 0$, function is convex(concave up)
  \item $f'' < 0$, function is concave(concave down)
  \item $f'' = 0$, $x_0$ is an inflection point
\end{itemize}

$f''$ also can tell there is no wiggle in graph

\textbf{How To Draw a Graph of Function}
\begin{enumerate}
  \item find discontinuities, especially when the value is infinite
  \item critical points, $f'(x) = 0$
  \item plot the zeros of $f$, $f(x) = 0$
  \item endpoints
  \item check local maximum/minimum, critical points and inflection points
\end{enumerate}

\textbf{Maximum and Minimum}

only exists in critical points, endpoints, or points of discontinuity

\subsection{Related Rates}
see \url{https://ocw.mit.edu/courses/mathematics/18-01-single-variable-calculus-fall-2006/lecture-notes/lec12.pdf}.

\subsection{Newton's Method}
Newton's method is a powerful tool for solving equations of the form $f(x) = 0$ by finding numerical approximations.

$$x_{k+1} = x_k - \frac{f(x_k)}{f'(x_k)}$$

Warning 1. Newton's Method can find an unexpected root.
Warning 2. Newton's Method can fail completely.

\subsection{Mean Value Theorem and Inequalities}
\textbf{Mean Value Theorem}

If $f$ is differentiable on $a < x < b$, and continuous on $a \le x \le b$, then
$$\frac{f(b)-f(a)}{b-a} = f'(c) (a<c<b)$$

\subsection{Differentials}
$dy = f'(x)dx$
\begin{exmp}
  solve $64.1^{\frac{1}{3}}$
  \begin{align*}
    & y = x^{\frac{1}{3}} \\
    & x = 64, y = 4, dx = 0.1 \\
    dy &\approx y'dx = \frac{1}{3}x^{-\frac{2}{3}dx} = \frac{1}{3}64^{-\frac{2}{3}} \times 0.1 \approx 0.002 \\
    64.1^{\frac{1}{3}} &\approx y+dy \approx 4+0.002 = 4.002
  \end{align*}
\end{exmp}

\textbf{Indefinite Integral}
$$F(x) = \int f(x)dx$$
$$F'(x) = f(x)$$

$$\int \sin x = -\cos x + c$$

\textbf{Substitution}

\begin{exmp}
  $\int \frac{1}{x\ln x}$ \\
  let u = $\ln x$, then $\int \frac{dx}{x\ln x} = \int \frac{1}{u}du = \ln u + c = \ln (\ln x) + c$
\end{exmp}

\textbf{Advanced Guessing}

\begin{exmp}
  $\int e^{6x}$ \\
  Guess $e^{6x}$ \\
  $$\frac{d}{dx}e^{6x} = 6e^{6x}$$ \\
  So $$\int e^{6x} = \frac{1}{6}e^{6x} + c$$ \\
\end{exmp}

\subsection{Differential Equations and Separation of Variables}
Using separation of variables.
$$\frac{dy}{dx} = f(x)g(y)$$
$$\frac{dy}{g(y)} = f(x)dx$$
$$H(y) = F(x) + c$$

\section{The Definite Integral and Its Applications}
\subsection{Definite Integrals}
Explanations
\begin{itemize}
  \item the area above the x axis minus the area below the x axis
  \item cumulative sum
  \item Riemann Integral: $\sum\limits_{i=1}^{n}f(c_i)\Delta x$
  \item $$\int_a^bf(x)dx = \lim_{x \rightarrow \infty} \sum_{i=1}^n f(a+\frac{i(b-a)}{n}\Delta x)$$
\end{itemize}

\subsection{First Fundamental Theorem of Calculus}
FTC1, Newton-Leibniz formula

If $f(x)$ is continuous and $F'(x) = f(x)$, then
$$\defint{a}{b}{f(x)} = F(b) - F(a) = F(x)\big|_a^b$$

\textbf{Intuitive Interpretation of FTC}

$x(t)$ is a position; $v(t) = x'(t)$ is the speed or rate of change of $x$.
$$\int_a^bv(t)dt = x(b) - x(a)$$

\textbf{Properties of Integrals}
\begin{enumerate}
  \item $\defint{a}{b}{(f(x) + g(x))} = \defint{a}{b}{f(x)} + \defint{a}{b}{g(x)}$
  \item $\defint{a}{b}{cf(x)} = c\defint{a}{b}{f(x)}$
  \item $\defint{a}{b}{f(x)} + \defint{b}{c}{f(x)} = \defint{a}{c}{f(x)}$
  \item $\defint{a}{a}{f(x)} = 0$
  \item $\defint{a}{b}{f(x)} = -\defint{b}{a}{f(x)}$
  \item if $f(x) \le g(x)$, then $\defint{a}{b}{f(x)} \le -\defint{a}{b}{g(x)}$ (for estimation)
\end{enumerate}

\textbf{Substitution of Integrals}

Only when $u'$ does not change sign.
$$\int_{x_1}^{x_2}f(x)dx = \int_{u_1}^{u_2}g(u)du \quad u_1 = u(x_1), u_2 = u(x_2)$$

\textbf{Another Explanation of FTC 1}
$$F(b) - F(a) = \int_a^bf(x)dx$$
let $\Delta F = F(b) - F(a)$,
$$\frac{\Delta F}{\Delta x} = \frac{1}{\Delta x}\int_a^bf(x)dx = \frac{1}{b-a}\int_a^bf(x)dx =Average(f) = Average(F')$$
So
$$\Delta F = Average(F')\Delta x$$
and
$$Average(f) = \frac{1}{n}\sum_{i=0}^nf(i) \approx \frac{1}{n}\int_0^nf(x)dx$$

$$\int_a^bmin(f)dx \le Average(F')\Delta x = \int_a^bf(x)dx \le \int_a^bmax(f)dx$$
\subsection{Second Fundamental Theorem of Calculus}
if $f$ is continuous and $G(x) = \int_a^xf(t)dt \quad (a \le t \le x)$, then $G'(x) = f(x)$ and $G(a) = 0$

\begin{exmp}
  \begin{align*}
    &\frac{d}{dx}\int_0^{x^2}\cos tdt = ? \\
    &u = x^2, F(u) = \int_0^u\cos tdt \\
    &\frac{d}{dx}\int_0^u\cos tdt = F'(u) = \cos u \\
    &\frac{d}{dx}\int_0^{x^2}\cos tdt = \frac{dF(x^2)}{dx} = F'(x^2)\cdot(x^2)' = 2x\cos x^2
  \end{align*}
\end{exmp}

\textbf{FTC2 VS MVT}
$$\Delta F = Ave(F'(x))\Delta x$$
$$\Delta F = F'(c)\Delta x$$

\textbf{New Functions/Transcendental Functions}

We can use integral to generate new functions, such as $\upperxdefint{2}{\frac{1}{lnt}}$.

If telling $L'(x) = \frac{1}{x}$, $L(1) = 0$, then $L(x) = \upperxdefint{1}{\frac{1}{t}}$.

\subsection{Applications to Logarithms and Geometry}
\textbf{Logarithm}

Regard $L(x) = \upperxdefint{1}{t}$ as the definition of the logarithm, then we have
$$L'(x) = \frac{1}{x}$$
$$L(1) = 0$$
$$L''(x) = -\frac{1}{x^2} < 0$$

\textbf{Areas between two curves}
$$A = \defint{a}{b}{(f(x) - g(x))}$$

\subsection{Volumes by Disks and Shells}
\textbf{Disks}

$y = -x^2+1 \quad (-1 \le x \le 1)$ rotated around the x-axis, calculate the volume.

\textbf{Shells}

$y = x^2$ rotated around the y-axis, calculate the volume.

\subsection{Work, Average Value, Probability}
\textbf{Continuous Average}

Riemann Sum
$$\frac{(y_1+y_2+\cdots+y_n)\Delta x}{b-a} \rightarrow \frac{\defint{a}{b}{f(x)}}{b-a}$$

\textbf{Weighted Average}

$$\frac{\defint{a}{b}{f(x)w(x)}}{\defint{a}{b}{w(x)}}$$

\textbf{Probability}

if
$$
f(x)=
\begin{cases}
0 \\
1
\end{cases}
$$

then, weighted average function becomes the probability.

$$P(x_1 \le x \le x_2) = \frac{\defint{x_1}{x_2}{w(x)}}{\defint{a}{b}{w(x)}} \quad (a \le x_1 \le x_2 \le b)$$

$$\frac{1}{\sqrt{\pi}}\defint{-\infty}{\infty}{e^{-x^2}} = 1$$

\subsection{Numerical Integration}
Numerical Integration is a way to compute an approximate solution to a definite integral.
\begin{enumerate}
  \item Riemann Sum: Left Riemann: $\sum\limits_{n=0}^{n-1}y_n\Delta x$, Right Riemann: $\sum\limits_{n=1}^{n}y_n\Delta x$
  \item Trapezoidal Rule: $\sum\limits_{n=0}^{n-1}\frac{y_n+y_{n+1}}{2}\Delta x = \frac{Left Riemann+ Right Riemann}{2}$
  \item Simpson’s Rule: $n$ is even, use parabola to calculate. $\sum\limits_{n=0}^{n-2}\frac{y_n+4y_{n+1}+y_{n+2}}{6}\Delta x$
\end{enumerate}

\section{Techniques of Integration}
\subsection{Trig Substitutions and Trig Integrals}

Solving $\int\sin^mxcos^nxdx$

\begin{itemize}
  \item Either $m$ or $n$ is odd, use $\sin^2x + \cos^2x = 1$ to substite to the result of only $\sin$ or $\cos$ exists, if $m$ is odd, let $u = \cos x$; if $n$ is odd, let $u = \sin x$.
  \item if both $m$ and $n$ are even, use double-angle formulae to depress the expression.
\end{itemize}

\subsection{Trig Substitution Rule}
\begin{itemize}
  \item $\sqrt{a^2-x^2} \rightarrow x = a\sin \theta \rightarrow result = a\cos \theta$
  \item $\sqrt{a^2+x^2} \rightarrow x = a\tan \theta \rightarrow result = a\sec \theta$
  \item $\sqrt{x^2-a^2} \rightarrow x = a\sec \theta \rightarrow result = a\tan \theta$
  \item $\sqrt{x^2+4x} = \sqrt{(x+2)^2-4} = \sqrt{(2\sec \theta)^2-4} = 2\tan \theta$
\end{itemize}

If necessarily, finally you should undo trig substitution by drawing a triangle.

\subsection{Partial Fractions}
Solving $\int\frac{P(x)}{Q(x)}$.

if degree P < degree Q, 
\begin{enumerate}
  \item factor the denominator
  \item set up equation, $\frac{A}{x+1} + \frac{B}{x+2}$
  \item solve A and B using cover-up method, $\frac{4x-1}{x+2} = A + \frac{B(x-1)}{x+2}$, $A = \frac{4-1}{1+2} = 1$
\end{enumerate}

if the equation has repeated roots, you should calculate the other variable first, and then plug them in to find the values that are relative to the repeated root.

if Q has a quadratic factor, calculate the other variable first ,then \textbf{clear the denominator}, finally plug them in to find the rest values.

if degree P $\ge$ degree Q, (improper fraction)
\begin{enumerate}
  \item use long division to find the quotient and the remainder
  \item $\frac{P(x)}{Q(x)} = quotient + \frac{R(x)}{Q(x)}$ where $R(x)$ is the remainder
  \item then $\frac{R(x)}{Q(x)}$ comes to the situation where degree P < degree Q
\end{enumerate}

\subsection{Integration by Parts}
\begin{gather*}
  (uv)' = u'v + uv' \\
  uv' = (uv)' - u'v \\
  uv' = uv - \int u'v
\end{gather*}

\subsection{Recurrence Formulas}
$$ \int(\ln x)^ndx = x(\ln x)^{n} - n\int (\ln x)^{n-1}dx$$

\subsection{Arc Length}

$$ds = \sqrt{(dx)^2 + (dy)^2} = \sqrt{1 + (\frac{dy}{dx})^2}$$

\subsection{Surface Area}
Sphere Surface Area, radius is $a$:
$$\int_{x_1}^{x_2} 2\pi yds = 2\pi a(x_2-x_1)$$

\subsection{Parametric Equations}
$$ x = a\cos t, b = a\sin t \rightarrow ds = adt$$
if changing speed, then it becomes $x = a\cos kt, b = a\sin kt$

ds in Parametric Equations
$$\frac{ds}{dt} = \sqrt{(\frac{dx}{dt})^2 + (\frac{dy}{dt})^2}$$

\subsection{Polar Co-ordinates, Area in Polar Co-ordinates}
$$Area = \pi r^2\frac{d\theta}{2\pi} = \frac{1}{2}r^2d\theta$$

\section{Exploring the Infinite}
\subsection{L‘Hospital's Rule}
Only suits \textbf{indeterminate form}, that is $\frac{0}{0}$ or $\frac{\infty}{\infty}$.
$$\limitx{a} \frac{f(x)}{g(x)} = \frac{f'(x)}{g'(x)}$$

\textbf{growing speed}
$$\ln x << x^p << e^x << e^{x^2}$$

$$\frac{1}{\ln x} >> \frac{1}{x^p} >> e^{-x} >> e^{-x^2}$$

\subsection{Improper Integrals}
$$\int_a^\infty f(x)dx = \lim_{M \rightarrow \infty f(x)dx}$$

the expression is converges if the limit exists, or else diverges.

\textbf{Important Integral}
$$\improperint{0}e^{-x^2}dx = \frac{\sqrt{\pi}}{2}$$

\textbf{Integral Comparison}

for powers,
$$\improperint{1}\frac{1}{x^p}dx$$
if $p \le 1$, diverges; if $p > 1$, limit = $\frac{1}{p-1}$

if $0 \le f(x) \le g(x)$,
\begin{itemize}
  \item if $\improperint{a}g(x)dx$ converges, so does $\improperint{a}f(x)dx$.
  \item if $\improperint{a}f(x)dx$ diverges, so does $\improperint{a}g(x)dx$.
\end{itemize}

\textbf{Improper Integrals of the Second Type}
$$\int_0^1\frac{1}{x^p}dx$$

if $p \le 1$, limit = $\frac{1}{p-1}$; if $p > 1$, diverges.

$$\int_0^1f(x)dx = lim_{a \rightarrow 0^+}\int_a^1f(x)dx$$

When a integral function contains singularity, it diverges.

\subsection{Infinite Series}
\textbf{Geometric Series}
$$1 + a + a^2 + \cdots = \frac{1}{1-a} \quad |a| < 1$$

$$S_N = \sum\limits_{i=0}^{N}a_i$$
When N goes infinite, if the limit of this \textbf{partial sum} exists, then the series \textbf{converges}.

\textbf{Integral Comparison}

Consider a positive, decreasing function $f(x) > 0$.
$$|\sum\limits_{n=1}^{\infty}f(n) - \improperint{1}f(x)dx| < f(1)$$

if $f(x) \sim g(x)$, then $\sum f(n)$ and $\sum g(n)$ eigher both converge or both diverge. \\
$f(x) \sim g(x)$ means
$$\limitx{\infty}\frac{f(x)}{g(x)} = c \quad (0 < c < \infty)$$

\textbf{Integral Test}

1. Limit Comparison
$$\sum\limits_{n=0}^{\infty}\frac{5n+2}{n^3+1} \sim \sum\limits_{n=0}^{\infty}\frac{1}{n^2}$$

2. Ratio Test
$$\lim_{n \rightarrow \infty}\frac{a_{n+1}}{a_{n}} = L$$
\begin{enumerate}
  \item if $L < 1$, $\sum a_n = L$ converges.
  \item if $L >1 $, $\sum a_n = L$ diverges.
  \item if $L = 1$, noting.
\end{enumerate}

\subsection{Taylor Series}
\textbf{Power Series}
$$f(x) = \sum_{n=0}^\infty a_nx^n = a_0 + a_1x + a_2x^2 + \cdots$$
$|x| < R$ where R = radius of convergence, means if $|x| > R$, then $|a_nx^n|$ does not tend to 0. if $a_n = c$, it becomes \textbf{geometric series}.

Rules of polynomials apply to series within the radius of convergence.
\begin{itemize}
  \item Substitution/Algebra
  \item Differentiation(term by term)
  \item Integration(term by term)
\end{itemize}

\textbf{Taylor's Series}
$$f(x) = f(b) + f'(b)(x-b) + \frac{f''(b)}{2}(x-b)^2 + \cdots $$

\end{document}
